\subsubsection*{Command formats}
In a command-line environment, GLUE commands are formed from simple strings of
text, entered interactively by the user, loaded from a batch file, or run from a
JavaScript program. In a web-based environment, commands may be sent as JSON
documents, attached to an HTTP request. Most commands are available in
both command-line or web-based environments.

A GLUE command invocation generates a result document: a structured data object
corresponding to a restricted form of JSON syntax. In web-based GLUE, the result
document is rendered as JSON and attached to the HTTP response. In command-line
GLUE the result document may be rendered to the console or saved to a file, in a
number of text-based formats. A special type of result document is the table
document, which may be rendered to tabular data formats.


In some cases, a GLUE-based resource may require business logic which is
specific to its usage scenarios or data requirements. For example the logic
which assembles the project data set may need to iterate over a certain set of
clades to process each associated tip alignment. Or a research project may
require analysis logic which extracts alignment rows from the data set and
computes a specific custom metric for each row.

GLUE project developers may write JavaScript programs, based on the ECMAScript
5.1 standard \cite{EcmaScript_2011} to address these requirements. The
JavaScript programs may invoke any GLUE command, and access the command results
as simple JavaScript objects. A simple example is given in figure
\ref{fig:scriptingLayer}.

  \begin{figure}[h!]
  \begin{center}
  \includegraphics[width=\linewidth]{figures/scriptingLayer/scriptingLayer.pdf}
  \end{center}
  \caption{\csentence{Example JavaScript program using the GLUE scripting
  layer.} The program iterates over a set of named tip alignments. It counts the
members of each alignment by invoking the \texttt{count member} command in the
appropriate command mode and stores the results in a JavaScript map. Finally
the map is output to the log.}
  \label{fig:scriptingLayer}
      \end{figure}

The GLUE scripting layer is based on the Nashorn scripting engine incorporated
into Java version 8. One advantage of this approach is that scripts are compiled
into Java bytecode and run within the Java virtual machine, and so benefit from
the memory management and other features of this environment.

The scripts may themselves be run by directly invoking the JavaScript file from
the GLUE command line. As an alternative, the script logic may be encapsulated
within a GLUE module of type \emph{ecmaFunctionInvoker}. In this case script
functions are exposed via the \texttt{invoke-function} module command. The logic
is stored within the database. Functions may return arbitrary GLUE command
documents and may be invoked as part of a web service, or from
higher-level scripts.
