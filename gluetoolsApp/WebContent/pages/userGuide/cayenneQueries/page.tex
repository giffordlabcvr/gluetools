\subsection*{Object-relational mapping}

Object-relational mapping (ORM) is a technique which allows object-oriented
programming languages such as Java to map between the objects, fields and
associations processed by application software on the other and the equivalent
tables, columns and relationships persisted in a relational database on the
other.

Internally, GLUE uses Apache Cayenne \cite{ApacheCayenne_2017} as its ORM
system. Apache Cayenne provides a simple but powerful expression language which
operates on objects, their fields and associations, but is translated internally
into the Structure Query Language (SQL) of the underlying database.

Many GLUE commands operate on sets of objects of the same type.
It is often useful to filter the set of objects by some user-defined criterion.
Consider for example, in \emph{Alignment} command mode, the following GLUE
command:

\begin{verbatim}
list member --recursive
\end{verbatim}

This lists the \emph{AlignmentMember} objects contained within the current
\emph{Alignment} or one of its descendents. Apache Cayenne expressions may be
supplied as ``where clause'' row selection options to many GLUE commands. This
will filter the set of objects on which the command operates. So, the above
example command may be extended as follows:

\begin{verbatim}
list member --recursive --whereClause "sequence.sequenceID like 'K%'"
\end{verbatim}

This will restrict the results to \emph{AlignmentMember} objects where the ID of
the associated \emph{Sequence} object begins with ``K''. The use of Apache Cayenne
to filter objects in this way provides significant expressive power as it is
able to traverse any relational link in the schema, without the user needing to
construct SQL syntax.