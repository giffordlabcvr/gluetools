
\subsubsection*{Command modes} GLUE commands are executed in a specific command
mode. Each command mode provides a different data object context and a specific
set of available commands. Command modes are arranged hierarchically, following
the object composition relationships within the GLUE schema.
Each command mode is represented by a mode path, specifying a particular data
object based on its location within the composition hierarchy.
Consider for example the following mode path:
\begin{center}
\begin{verbatim}
/project/hcv/reference/REF_MASTER_NC_004102/feature-location/NS5B
\end{verbatim}
\end{center}
This represents the command mode associated with the \texttt{NS5B}
\emph{FeatureLocation} within the \emph{ReferenceSequence} 
\texttt{REF\_MASTER\_NC\_004102}, itself contained within the GLUE project
\texttt{hcv}. In this mode, commands are available which operate on this
specific \emph{FeatureLocation} object, for example listing the
set of \emph{FeatureSegment} objects or creating a new
\emph{Variation} object contained within the \emph{FeatureLocation}. The mode
path mechanism allows commands which operate on the same object context
to be grouped together.

Command-line GLUE provides mechanisms for navigating between command modes. For
commands submitted to web-based GLUE, the mode path is used as the final part of
the request URL. 


The GLUE interactive command line system is a powerful environment focused on
the development of GLUE projects at a high level of productivity. The command line
design could be compared to interactive R or Python 
interpreters \cite{R_2017,Python_2017}, or an interactive database command line
client.

The interactive command line system has three main uses:
\begin{itemize}
  \item Testing the operation of GLUE commands that are later integrated
  permanently into a GLUE project
  \item Querying a project dataset to check that it has been assembled
  correctly
  \item Running data queries or other analysis functions for research purposes
\end{itemize} 

Productivity-oriented features include:
\begin{itemize}
  \item Automatic completion of both the keywords and arguments of partially-typed commands
  \item Interactive paging through tabular data
  \item A command history which is saved between GLUE sessions
  \item Line editing and history keystrokes similar to GNU Readline
\end{itemize}
