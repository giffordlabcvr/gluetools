
The core GLUE schema is present in every GLUE project and forms the basis for
the standard, built-in functionality. However, there are often important
auxiliary data items which are of interest in the target usage scenarios. These
project-specific data objects may have highly structured relationships with each
other and with objects in the core schema.

GLUE provides a powerful yet easy-to-use mechanism for extending the database
schema on a per-project basis. New fields may be added to tables in the core
schema. New custom tables may be added, with their own data fields.
Finally, custom relational links may be added between any pair tables in the
schema.

Schema extensions are defined using GLUE commands. Once schema extensions are
defined in a GLUE project, the custom tables, fields and links may be populated
with data either using fine-grained GLUE commands, or using certain GLUE modules
such as \emph{genbankXmlPopulator}. The custom tables, fields and links may then play
a role in programmatic row selection criteria or in any other GLUE functionality
which traverses the data schema.


The United Nations defines the M.49 system \cite{M49_1996} for classifying
countries and grouping them hierarchically into geographical regions. This
system can be incorporated into a GLUE project, as a schema extension. An
example rationale for this is to allow the comparison of virus sequences not
only between countries but between global regions. We give a brief outline
showing how a schema extension may operate; the full extension, which has been
used in the HCV-GLUE project, is more detailed.

Within the \texttt{schema-project} GLUE command mode, custom tables are
introduced for countries, regions and sub-regions:
\begin{verbatim}
create custom-table m49_country
create custom-table m49_region
create custom-table m49_sub_region
\end{verbatim}

Objects in custom tables use strings as their unique ID. For the
\texttt{m49\_country} table, the 3-letter ISO code may be adopted as the object
ID. Other fields associated with each country may then also be defined for
custom tables:

\begin{verbatim}
table m49_country
  create field m49_code INTEGER
  create field full_name VARCHAR 100
\end{verbatim}

Similarly, fields may also be defined for the \texttt{m49\_region} and
\texttt{m49\_sub\_region} tables. A country in the M.49 system is associated
with a single sub-region, but a sub-region may contain many countries. This may
be modelled by introducing a link into the schema extension:

\begin{verbatim}
create link m49_country m49_sub_region --multiplicity MANY_TO_ONE
\end{verbatim}

The association between sequences and countries may also be modelled by adding a
link definition to the schema, in this case between the core \emph{Sequence}
table and the custom table \texttt{m49\_country}.

\begin{verbatim}
create link sequence m49_country --multiplicity MANY_TO_ONE
\end{verbatim}

Data items may now be added, making use of these schema extensions. For example,
an object may be created to represent France (M.49 code 250, ISO code FRA), and
associated with the object representing the M.49 sub-region for Western Europe
via the following commands in \texttt{project} mode:

\begin{verbatim}
create custom-table-row m49_country FRA
custom-table-row m49_country FRA
  set field m49_code 250
  set field full_name France
  set link-target m49_sub_region western_europe
\end{verbatim}

A \emph{Sequence} object may be associated with its country of origin by
using the \texttt{set link-target} command in the appropriate command mode:

\begin{verbatim}
sequence ncbi-fmdv AY593780
  set link-target m49_country FRA
\end{verbatim}

Such custom extensions may then be used in row selection criteria, for example
this \texttt{list sequence} command will list all those sequences associated
with a country in Western or Northern Europe:

\begin{verbatim}
list sequence --whereClause "m49_country.m49_sub_region.id in
('western_europe', 'northern_europe')"
\end{verbatim}

