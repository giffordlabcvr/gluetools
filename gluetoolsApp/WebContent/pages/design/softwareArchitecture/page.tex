While GLUE is implemented in the Java programming language, its command layer
consumes and produces structured document objects which are independent of any
programming language. These documents may take the form of scripting text,
eXtensible Markup Language (XML), JavaScript Object Notation (JSON) or tabular
formats where appropriate. The multi-format, language-independent nature of the
command layer allows GLUE to be integrated quickly into a broad variety of
computing contexts.

GLUE can be installed on computers running the Microsoft Windows, Apple OSX, or
Linux operating systems. There are two alternative deployment patterns:
\begin{enumerate}
\def\labelenumi{\arabic{enumi}.}
\item Command-line GLUE, typically installed on a private server or desktop
  computer, is suitable for project development and more advanced
  bioinformatics. In such a deployment, the interactive GLUE command line may be
  used, or GLUE may be run non-interactively based on GLUE batch files or
  JavaScript progams.
\item Web-based GLUE, installed on a public server or privately as part of a
  microservices architecture, for GLUE projects with a web UI or
  programmatic web API.
\end{enumerate}


The analysis of virus sequence data requires that different kinds of information
are assembled together. For example, to apply drug resistance analysis to a new
hepatitis C sequence, it must be assigned to a genotype, therefore genotype
definitions and any data forming part of the assignment process are required for
this step. The reading frames and protein translations of the drug target genes
in the new sequence must be determined, for example via comparison with an
annotated reference sequence. Therefore this reference sequence and its
annotations are also required. Finally, the scenario requires the set of
polymorphisms known to be associated with drug resistance so that these may be
detected in the new sequence.

