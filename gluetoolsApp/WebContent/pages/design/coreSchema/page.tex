As an example from hepatitis C research, a nucleotide binding motif (NBM) was
suggested in the NS4B protein \cite{Einav_2004}. The most conserved element of
the NBM is the `A' motif consisting of a Glycine (G) at codon position 129
followed by four amino acids of any type, followed by a Glycine (G) at position
134 and a Lysine (K) at position 135. The research showed that Arginine (R) or
Serine (S) could be substituted for the Lysine with only a small reduction in
the binding activity, so one formulation of the `A' motif might allow either of
these substitutions.

This can be represented using a \emph{Variation} containing a single
\emph{PatternLocation}, defined within NS4B of the HCV master
\emph{ReferenceSequence} which is the basis for standardised HCV codon
coordinates \cite{Kuiken_2006}. Figure \ref{fig:variationExample} shows the GLUE
commands to construct such a variation within the HCV-GLUE project.

  \begin{figure}[h!]
  \begin{center}
  \includegraphics[width=\linewidth]{figures/variationExample/variationExample.pdf}
  \end{center}
  \caption{\csentence{GLUE commands to create a \emph{Variation} representing
  the NBM `A' motif in HCV NS4B.} The object is created within
  the NS4B \emph{FeatureLocation} of the master \emph{ReferenceSequence}
  allowing standard codon coordinates to define the location. The pattern
  itself is specified using a regular expression, in which .{\{4\}} allows for a
  sequence of any four residues, and [KRS] allows either Lysine, Arginine or
  Serine. }
  \label{fig:variationExample}
      \end{figure}

      
  reference REF_MASTER_NC_004102
  feature-location NS4B
    create variation NBM_A --translationType AMINO_ACID
    variation NBM_A
      create pattern-location "G.{4}G[KRS]" --labeledCodon 129 135
      exit
    exit
  exit
  